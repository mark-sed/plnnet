\documentclass[a4paper, 11pt]{article}
%\usepackage[utf8x]{inputenc}
\usepackage[total={18cm,25cm}, top=2cm, left=1.5cm, includefoot]{geometry}
\usepackage[czech]{babel}
\usepackage{times}
\usepackage[T1]{fontenc}
\usepackage[utf8]{inputenc}
\usepackage{hyperref}
\usepackage{amssymb}
\usepackage{amsmath, amsthm}
\usepackage{mathtools, moresize}
\usepackage{amsmath}
\usepackage{float}
\usepackage{ragged2e}
\usepackage{graphicx} 
\usepackage{pdfpages}
\usepackage{icomma}
\usepackage{algorithm2e}
\usepackage{listings}
\usepackage{nccmath}



\usepackage[
%backend=biber,
style=numeric,
sorting=ynt,
  backend=bibtex      % biber or bibtex
%,style=authoryear    % Alphabeticalsch
 ,sortcites=true      % some other example options ...
 ,block=none
 ,indexing=false
 ,citereset=none
 ,isbn=true
 ,url=true
 ,doi=true            % prints doi
 ,natbib=true         % if you need natbib functions
]{biblatex}
 
\addbibresource{literatura.bib}

\providecommand{\uv}[1]{\quotedblbase #1\textquotedblleft}

\title{Prolog Neural Network\\\Large Funkcionální a logické programování\,--\,2. projekt}
\date{}
\author{Bc. Marek Sedláček \texttt{(xsedla1b)}}

\begin{document}
\maketitle

\section{Prolog Neural Network (PLNNet)}

PLNNet je projekt na ukázku implementace jednoduché neuronové sítě v jazyce Prolog (SWI-Prolog). Funkce pracují zásadně s maticemi pro ulehčení programování a projekt obsahuje ukázkové a interaktivní funkce.

\section{Implementace}

Projekt je rozdělený do 3 modulů, a to: \texttt{input2}, \texttt{plnnet} a \texttt{nn}. 

\begin{itemize}
    \item Modul \texttt{input2} je lehce modifikovaná verze skriptu se stejným jménem poskytnutá k tvorbě projektu.
    \item Modul \texttt{nn} obsahuje funkce pro práci s maticemi, soubory dat pro trénování a představuje neuronovou síť samotnou. Funkce a operátory pro práci s maticemi jsou z modulu exportovány a je tedy možné je snadno využít pro případné úpravy vstupních nebo výstupních matic.
    \item Modul \texttt{plnnet} je ukázkový kód práce s modulem \texttt{nn} včetně interaktivní funkce pro natrénování neuronové sítě a následné predikce dle vstupů uživatele. 
\end{itemize}

\subsection{Proces trénování a predikce}

Před výpočtem predikce dle uživatelského vstupu je potřeba provést trénování, což je možné pomocí funkce \texttt{training/6}, která jako argumenty přijímá aktivační funkci, trénovací vstupy, očekávané výstupy, váhy, počet epoch na trénování a~jejím výstupem jsou nově natrénované váhy pro jednotlivé iterace trénování.

Aktivačních funkcí nabízí modul \texttt{nn} více:
\begin{itemize}
    \item \texttt{sigmoid/3} -- sigmoida,
    \item \texttt{tanh/3} -- hyperbolická funkce,
    \item \texttt{softsign/3} -- funkce softsign,
    \item \texttt{gaussian/3} -- Gaussova funkce.
\end{itemize}

\noindent Je-li jakákoliv funkce volána s hodnotou \texttt{true} jako druhým argumentem, vrátí tato funkce svou derivaci, což je potřeba při trénování v procesu backtrackingu.

Funkce \texttt{training/6} provádí volání forward funkce následované backward funkcí pro všechny iterace.

Po natrénování vah je možné zavolat predikční funkci \texttt{predict/6} opět s aktivační funkcí, vstupem pro predikci a natrénovanými váhami. Výstupem je predikce pro vstupní hodnoty.

\section{Spuštění}

Pro spuštění programu stačí pouze interpretu \texttt{swipl} jako první argument zadat soubor \texttt{plnnet.pl}:

\begin{verbatim}
    swipl plnnet.pl
\end{verbatim}

\noindent Nebo pomocí Makefile s možností \texttt{interactive}:

\begin{verbatim}
    make interactive
\end{verbatim}

Tímto se spustí interaktivní textové rozhraní, kde stačí zadat počet epoch pro trénování a následně data na predikci (oddělené mezerou).

Další možností je překlad do binárního souboru pomocí Makefile (příkaz \texttt{make}). Tato binární verze však volá pouze funkci \texttt{main} a není tedy možné zde provádět testování a predikce bez trénování (pro tento přístup je vhodný \texttt{interactive} mód).

Trénovací data jsou načtená ze souboru \texttt{train}, výstupy pro jednotlivá data ze souboru \texttt{outputs} a váhy ze souboru \texttt{weights}. Kde formát těchto souborů vždy obsahuje na jednom řádku jedny vstupní data obalená hranatými závorkami, jednotlivé hodnoty jsou odděleny čárkou a řádek je ukončen tečkou. Tedy například:

\begin{verbatim}
    [4.2, 3.14, 0.1].
    [1.1, 5.42, 8.4].
\end{verbatim}

\noindent Tento soubor by reprezentoval matici:
\\

\begin{fleqn}
 \makebox{} \left| \begin{array}{ccc}
4,2 & 3,14 & 0,1 \\
1,1 & 5,42 & 8,4 \end{array}\right| 
\end{fleqn}
\\

Pro provedení predikce je tato hodnota vypsána a natrénované váhy jsou uloženy do predikátu \texttt{trainedWeights/1}, pro případné další predikce, které je možné provést například následovně:

\begin{verbatim}
    trainedWeights(W), predict(sigmoid, [[1.0, 0.0, 0.0]], W, P).
\end{verbatim}

\noindent Případně je možné zavolat funkci \texttt{main/0}, která spustí opět interaktivní rozhraní.

\subsection{Výchozí neuronová síť}

Výchozí nastavení vstupních souborů neuronové sítě jsou data, kde se má síť naučit, že očekávaným výstupem predikce je první hodnota ve vstupu booleanových hodnot zapsaných v pohyblivé řádové čárce podobě (pro \texttt{[[1.0 0.0 0.0]]} je to \texttt{[[1.0]]}). 

\subsection{Testování sítě}

Pro testování sítě na výchozí vstupní data je možné zavolat funkci \texttt{runTests/1} s počtem epoch na natrénování a tato funkce poté načte testovací data ze souboru \texttt{test} a vypíše pro jednotlivé vstupy očekávanou predikci (první hodnotu) a skutečnou predikci.


%\vspace{10em}

%\nocite{paral_alg}
%\nocite{mpi_man}
%\printbibliography[title=Literatura,heading=bibintoc]
\end{document}
